\documentclass{report}

\usepackage{polski} % Pozwala na użycie polskiego. Ustawia między innymi fontenc na T1
\usepackage[utf8]{inputenc} % Informuje o kodowaniu
\usepackage{enumitem}
\usepackage{xcolor}
\usepackage{xcolor}% http://ctan.org/pkg/xcolor
\usepackage{hyperref}% http://ctan.org/pkg/hyperref

\definecolor{LinkColor}{HTML}{1d5cc1}

\usepackage{tabto}

\usepackage{graphicx} % Pakiet do obrazów
\graphicspath{ {./Obrazy/} } % Folder, z którego będą brane obrazy

% Nie twórz nowych stron
\usepackage{etoolbox}
\makeatletter
% \patchcmd{\chapter}{\if@openright\cleardoublepage\else\clearpage\fi}{}{}{}
\makeatother

\title{Specyfikacja implementacyjna -- Gra w życie}
\author{Krzysztof Dąbrowski i Jakub Bogusz}
\date{\today}

\begin{document}
\maketitle{}

\tableofcontents{}

\chapter{Opis klas}

\section{Package ,,Models''}
Package składający się z klas reprezentujących odpowiednie automaty komórkowe, odpowiedzialnych za przechowywanie ich zasad, przeprowadzanie symulacji i generowanie kolejnych pokoleń. Będzie on zawierać 3 klasy, jedną ogólną \texttt{CellularAutomaton}, łączącą w sobie cechy wspólne wszystkich automatów komórkowych oraz 2 klasy dziedziczące z poprzedniej, opisujące działanie konkretnych automatów (\texttt{GameOfLife} oraz \texttt{WireWorld}).

\subsection{CellularAutomaton}



\end{document}
